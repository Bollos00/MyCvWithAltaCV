%%%%%%%%%%%%%%%%%
% This is an sample CV template created using altacv.cls
% (v1.6.5, 3 Nov 2022) written by LianTze Lim (liantze@gmail.com). Compiles with pdfLaTeX, XeLaTeX and LuaLaTeX.
%
%% It may be distributed and/or modified under the
%% conditions of the LaTeX Project Public License, either version 1.3
%% of this license or (at your option) any later version.
%% The latest version of this license is in
%%    http://www.latex-project.org/lppl.txt
%% and version 1.3 or later is part of all distributions of LaTeX
%% version 2003/12/01 or later.
%%%%%%%%%%%%%%%%

%% Use the "normalphoto" option if you want a normal photo instead of cropped to a circle
% \documentclass[10pt,a4paper,normalphoto]{altacv}

\documentclass[11pt,a4paper,ragged2e,withhyper]{altacv}
%% AltaCV uses the fontawesome5 and packages.
%% See http://texdoc.net/pkg/fontawesome5 for full list of symbols.

% Change the page layout if you need to
\geometry{left=1.25cm,right=1.25cm,top=1.5cm,bottom=1.5cm,columnsep=1.2cm}

% The paracol package lets you typeset columns of text in parallel
\usepackage{paracol}

% Change the font if you want to, depending on whether
% you're using pdflatex or xelatex/lualatex
\ifxetexorluatex
  % If using xelatex or lualatex:
  \setmainfont{Roboto Slab}
  \setsansfont{Lato}
  \renewcommand{\familydefault}{\sfdefault}
\else
  % If using pdflatex:
  \usepackage[rm]{roboto}
  \usepackage[defaultsans]{lato}
  % \usepackage{sourcesanspro}
  \renewcommand{\familydefault}{\sfdefault}
\fi

% Change the colours if you want to
\definecolor{SlateGrey}{HTML}{2E2E2E}
\definecolor{LightGrey}{HTML}{666666}
\definecolor{DarkBlue}{HTML}{000044}
\definecolor{Blue}{HTML}{0000AA}
\colorlet{name}{black}
\colorlet{tagline}{Blue}
\colorlet{heading}{DarkBlue}
\colorlet{headingrule}{black}
\colorlet{subheading}{Blue}
\colorlet{accent}{Blue}
\colorlet{emphasis}{SlateGrey}
\colorlet{body}{LightGrey}

% Change some fonts, if necessary
\renewcommand{\namefont}{\Huge\rmfamily\bfseries}
\renewcommand{\personalinfofont}{\footnotesize}
\renewcommand{\cvsectionfont}{\LARGE\rmfamily\bfseries}
\renewcommand{\cvsubsectionfont}{\large\bfseries}


% Change the bullets for itemize and rating marker
% for \cvskill if you want to
\renewcommand{\itemmarker}{{\small\textbullet}}
\renewcommand{\ratingmarker}{\faCircle}

%% Use (and optionally edit if necessary) this .tex if you
%% want to use an author-year reference style like APA(6)
%% for your publication list
% \input{pubs-authoryear.cfg}

%% Use (and optionally edit if necessary) this .tex if you
%% want an originally numerical reference style like IEEE
%% for your publication list
\input{pubs-num.cfg}

\addbibresource{cv.bib}

\begin{document}
\name{Bruno Bollos Correa}
\tagline{Engenheiro Eletricista | Telecomunicações | Software}
%% You can add multiple photos on the left or right
\photoR{2.8cm}{foto_linkedin.jpg}
% \photoL{2.5cm}{Yacht_High,Suitcase_High}

\personalinfo{%
  % Not all of these are required!
  \email{bollos@outlook.com}
  % \phone{000-00-0000}
  % \mailaddress{Åddrésş, Street, 00000 Cóuntry}
  \location{São Paulo-SP, Brasil}
  % \homepage{www.homepage.com}
  % \twitter{@twitterhandle}
  \linkedin{bollos00}
  \github{Bollos00}
  \NewInfoField{gitlab}{\faGitlab}[https://gitlab.com/]
  \gitlab{Bollos00}

  % \orcid{0000-0000-0000-0000}
  %% You can add your own arbitrary detail with
  %% \printinfo{symbol}{detail}[optional hyperlink prefix]
  % \printinfo{\faPaw}{Hey ho!}[https://example.com/]
  %% Or you can declare your own field with
  %% \NewInfoFiled{fieldname}{symbol}[optional hyperlink prefix] and use it:
  % \NewInfoField{gitlab}{\faGitlab}[https://gitlab.com/]
  % \gitlab{your_id}
  %%
  %% For services and platforms like Mastodon where there isn't a
  %% straightforward relation between the user ID/nickname and the hyperlink,
  %% you can use \printinfo directly e.g.
  % \printinfo{\faMastodon}{@username@instace}[https://instance.url/@username]
  %% But if you absolutely want to create new dedicated info fields for
  %% such platforms, then use \NewInfoField* with a star:
  % \NewInfoField*{mastodon}{\faMastodon}
  %% then you can use \mastodon, with TWO arguments where the 2nd argument is
  %% the full hyperlink.
  % \mastodon{@username@instance}{https://instance.url/@username}
}

\makecvheader
%% Depending on your tastes, you may want to make fonts of itemize environments slightly smaller
% \AtBeginEnvironment{itemize}{\small}

%% Set the left/right column width ratio to 6:4.
\columnratio{0.5}

% Start a 2-column paracol. Both the left and right columns will automatically
% break across pages if things get too long.
\begin{paracol}{2}
\cvsection{Experiência}

\cvevent{Desenvolvedor de Software}{Equipe de Robótica da FEI (RoboFEI)}{Abril/2019 -- Novembro/2022}{}
\begin{itemize}
\item Habilidades: Desenvolvimento de software, C++, Qt, Git, robótica, trabalho em equipe.
\item \href{https://gitlab.com/robofei/ssl}{Repositórios de código aberto no GitLab}
\item Participação em 5 competições, sendo 3 nacionais e 2 internacionais.
\end{itemize}

\cvsection{Projetos}

\cvevent{Trabalho de Iniciação Científica}{Centro Universitário da FEI/RoboFEI}{Agosto/2021 -- Julho/2022}{}
\begin{itemize}
\item Análise de técnicas de aprendizado supervisionado aplicadas na avaliação de sucesso de ações no futebol de robôs.
\item Orientação de Prof. Dr. Flavio Tonidandel.
\item Habilidades: Pesquisa, C++, Qt, Python, Aprendizado de Máquina, scikit-learn.
\end{itemize}

\divider

\cvevent{Trabalho de Conclusão de Curso}{Centro Universitário da FEI e Companhia do Metrô de São Paulo}{Fevereiro/2022 -- Dezembro/2022}{}
\begin{itemize}
  \item Planejador de rotas com navegação indoor no sistema metroferroviário de São Paulo.
  \item Orientação de Prof. Dr. Marco Antonio Assis de Melo e Eng. Felipe Copche.
  \item Habilidades: Pesquisa, Python, FastAPI, PostgreSQL, Neo4j, Docker, Java, Android, Bluetooth, embarcados (ESP32).
\end{itemize}
  
\medskip

\cvsection{Publicações}

%% Specify your last name(s) and first name(s) as given in the .bib to automatically bold your own name in the publications list.
%% One caveat: You need to write \bibnamedelima where there's a space in your name for this to work properly; or write \bibnamedelimi if you use initials in the .bib
%% You can specify multiple names, especially if you have changed your name or if you need to highlight multiple authors.
% \mynames{Lim/Lian\bibnamedelima Tze,
%   Wong/Lian\bibnamedelima Tze,
%   Lim/Tracy,
%   Lim/L.\bibnamedelimi T.}
%% MAKE SURE THERE IS NO SPACE AFTER THE FINAL NAME IN YOUR \mynames LIST

\nocite{*}
% \printbibliography
% \printbibliography[heading=pubtype,title={\printinfo{\faBook}{Books}},type=book]

% \divider

\printbibliography[heading=pubtype,title={\printinfo{\faFile*[regular]}{Artigos Acadêmicos}},type=article]

% \divider

% \printbibliography[heading=pubtype,title={\printinfo{\faUsers}{Conference Proceedings}},type=inproceedings]

%% Switch to the right column. This will now automatically move to the second
%% page if the content is too long.
\switchcolumn

\cvsection{Habilidades}

\begin{itemize}
  \item Robótica
  \item Software
  \item Embarcados
  \item Pesquisa
\end{itemize}

\divider

Conhecimento e experiência em tecnologias de software:
\begin{description}
  \item[C++, Qt, Git] 4 anos
  \item[Python] 3 anos
  \item[Docker] 2 anos
  \item[FastAPI, Neo4j, Java, Android] 6 meses
\end{description}

\cvsection{Idiomas}

\cvskill{Português}{5}
\divider

\cvskill{Inglês}{4}
\divider

\cvskill{Espanhol}{3}

\medskip

\cvsection{Educação}

\cvevent{Graduação em Engenharia Elétrica com ênfase em Telecomunicações}{Centro Universitário da FEI}{2018--2022}{}

\cvsection{Referências}

% \cvref{name}{email}{mailing address}
\cvref{Prof. Dr. Marco Antonio Assis de Melo}{Centro Universitário da FEI}{mant@fei.edu.br}{}

\end{paracol}

\end{document}
